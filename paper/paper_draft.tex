%! Author = natem
%! Date = 2022-02-24

% Preamble
\documentclass[12pt, a4paper]{article}


% Packages
\usepackage[utf8]{inputenc}
\usepackage{authblk}
\usepackage{hyperref}
\usepackage[a4paper, margin=1in]{geometry}

% Styling
\renewcommand{\familydefault}{\sfdefault}
\title{GENOMIC SURVEILLANCE REVEALS HIGHLY CONNECTED SARS-CoV-2 EPIDEMICS}
\author[1]{Nathaniel L. Matteson* \\ \href{mailto:natem@scripps.edu}{natem@scripps.edu}}
\affil[1]{Department of Immunology and Microbiology, Scripps Research }
\date{}


% Document
\begin{document}
    \maketitle

    \begin{abstract}
        Collectively, the San Ysidro and Otay Mesa border crossings between San Diego and Baja California are the busiest border crossings in the Americas and the second busiest in the world.
        With an estimated 4 million travelers a month before the pandemic, the border potentially represents a sizable avenue for the spread of SARS-CoV-2 and an important consideration for the implementation of international travel restrictions.
        While previous studies have identified variability in the frequency of importation events and the amount of transmission lineage persistence across international borders, few have looked specifically at the US-Mexico border, where large population centers straddle the border and form closely connected, binational communities.
        In this study, we investigate the relative contribution of cross-border transmissions versus local transmission on epidemics in both border communities.
        We find that lineages from both locations share a significant amount of evolutionary history and we attribute this finding to an abundance of cross-border transmission resulting from the highly mobile population situated at the border.
        Transmission lineages that originated across the border were not significantly different from those that originated from domestic sources, however, we still observed that they had a significant effect on cases in both San Diego and Baja California, particularly during the early epidemic.
        Our findings highlight the complex meaning of ‘local’ when investigating highly-connected epidemics and emphasize the importance of collaborative interventions when such situations are expected.
    \end{abstract}

    \newpage

    \raggedright
    \section*{Introduction}

    \section*{Results}
    \subsection*{Amount of shared evolutionary history is associated with proximity}
    \subsection*{Similar magnitude of jumps into San Diego from Los Angeles as from Baja California}
    \subsection*{Cross-border transmissions appear similar to domestic transmission lineages}
    \subsection*{Relative importation risk into San Diego changed significantly over the first two years of pandemics}
    \subsection*{Cross-border transmission lineages contribute significantly to transmission in San Diego}
    \subsection*{Differing highly mobile populations in the US and Mexico drive cross-border transmissions}
    \subsection*{Highly-mobile populations drive cross-border transmissions}
    \subsection*{Enacted border closure had a small effect on cross-border transmission}
    \subsection*{More restrictive border closures prevent epidemic connectivity}
    \section*{Discussion}

    \newpage
    \section*{Methods}
    \paragraph{SARS-CoV-2 Amplicon Sequencing}
    SARS-CoV-2 was sequenced using PrimalSeq-Nextera XT\@.
    This protocol is based on the ARTIC PrimalSeq protocol, except that amplicon sizes were reduced to enable 2x150 read length requirements.
    The ARTIC network nCoV-2019 V4 primer scheme uses two multiplexed primer pools to create overlapping 250 bp amplicon fragments in two PCR reactions.
    Full details of protocol can be found here: \url{https://andersen-lab.com/secrets/protocols/}.
    Briefly, SARS-CoV-2 RNA (2 mL) was reverse transcribed with LunaScript RT (x).
    The virus cDNA was amplified in two multiplexed PCR reactions (one reaction per ARTIC network primer pool) using Q5 DNA High-fidelity Polymerase (New England Biolabs).
    Following an AMPureXP bead (Beckman Coulter) purification of the combined PCR products, sequencing adaptors containg sample specific indexes were added using a step-out PCR reaction using Q5 DNA High-fidelity Polymerase.
    The libraries were purified with AMPureXP beads and quantified using the Qubit High Sensitivity DNA assay kit (Invitrogen) and Tapestation D5000 tape (Agilent).
    The individual libraries were normalized and pooled in equimolar amounts at 2 nM\@.
    The 2 nM library pool was sequenced on an Illumina NovaSeq 6000 using a Xxx kit.
    \paragraph{SARS-CoV-2 Genomic Data}

    \paragraph{Phylosor Analysis}
    We used the global SARS-CoV-2 phylogeny provided by COG.UK as of April 2022.
    For each location pair, the phylogeny was pruned to sequences present in the above dataset that were collected from either location.
    Using the pruned tree, for each month in the period of January 2020-July 2021, the phylosor metric was calculated using only sequences collected in that month.
    Briefly, the phylosor metric is calculated as the ratio of branch lengths (in units of per-site mutation rate) that are shared by two sets of tips (\(BL_{Both}\)) compared to the total branch length that is unique to each set of tips.
    \[
        \frac{BL_{Both}}{(BL_A + BL_B) * 0.5}
    \]
    Where \(BL_A\) and \(BL_B\) indicate the total branch lengths of either set A or B\@.
    \\~\\
    To compare the observed similarity with the maximum possible similarity given the pruned tree topology, locations states were shuffled within each epiweek and phylosor was recalculated as described above.
    This was performed ten times for each location pair.
    The difference between the observed phylosor measurement and those performed on the "mixed model" was deemed phylosor distance.

    \paragraph{Phylogeny Simulation}
    To validate the use of phylosor in measuring the temporal connectivity between locations, we conductic epidemic simulations using FAVITES V1.1.35.
    First we generated static contact networks in FAVITES using a modified Barabási-Albert algorithm.
    We generated three seperate 20,000 member communities using the Barabási-Albert algorithm with a mean value of 8 contacts per day.
    For each community, we calculated intra-community connectivity as the fraction of all possible contacts that were made.
    Inter-community edges were sampled by randomly deciding for each pair of nodes in different communities if they should be connected by an edge or not.
    The probability of connected two nodes in two communities was calculated as a fraction of the average intra-community connectivity.
    For our purposes, we specified two communities as having a connectivity \(1/4\) of the intra-communitity connectivity, and the third community have a connectivity of \(1/16\) the intra-community connectivity to the other communities.
    \\~\\
    We then simulated a transmission network over the contact network using a Susceptible-Infected-Recovered model.
    A single viral lineage was sampled at a random point from each infected indiivudual during their infection time to represent viral genome sequencing.
    The virus phylogeny in units of time (years) was sampled under a constant coalescnet model using the Virus Tree Simulator package embedded in FAVITES.
    All parameters for the transmission network simulation and viral lineage sample were consistent with Worobey et al. 2020.
    Ultimately, the virus phylogeny in units of time was converted to units of per-site mutation rate by multiplying the branch length by a constant 0.0008 subs/site/year, consistent with Xxx et al.
    We produced 10 simulations for each condition.

    \paragraph{Genomic dataset generation}
    The massive amount of sequencing data produced during the COVID-19 pandemic prevent us from constructing an un-downsampled phylogeny.

    In order to limit the computational burden of the phylogeographic analysis, we subsampled 2500 sequences from all available sequences on GISAID for the period up to July 31st, 2021.
    To focus the analysis in on the region around San Diego county, we allocated 500 sequences each to San Diego county, Los Angeles county, and Baja California.
    \\~\\
    The remaining 1000 sequences were allocated to all other locations proportionally to their distance to San Diego and the total number of flights connecting the location and San Diego in 2019.
    Here, locations refers to the state (or first administration level) in the US and Mexico, and country everywhere else.
    Geographic distance was calculated as the centroid-centroid distance to San Diego county, rescaled to have a unit range, and inverted, so that nearby locations had the greatest weight.
    Total number of flights into San Diego was also rescaled to have a unit range.
    The proportion of the sum of these two weights relative to all other locations was the proportion of the 1000 contextual sequences allocated to that location.
    In order to sample virus diversity in each location equally, sequences were randomly sampled proportional to the location-specific incidence data binned by epidemiological week.
    \\~\\
    To include a reasonable root in our phylogenetic inference, we also include the 50 earliest SARS-CoV-2 genomes in our dataset.
    Lastly, to assess the accuracy of the timing of the basal structure of our phylogeny, we included genomes from three outbreaks with well-described introductions.
    A list of all included sequences, their GISAID accession IDs, and the compartment they filled is shown in Table X.

    \paragraph{Phylogenetic Analyses}
    \paragraph{Epidemiological Data and the Estimation of Infection Rate}
    \paragraph{Travel and Mobility Data}
    \paragraph{Estimated Importation Risk}
    \paragraph{Contact tracing data}


\end{document}